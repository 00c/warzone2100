\documentclass[12pt]{article}
\usepackage{listings}
\usepackage{underscore}
\lstloadlanguages{C++}
\lstset{language=C++, frame=trBL, aboveskip=15pt, belowskip=15pt, xleftmargin=20pt, xrightmargin=20pt}
\title{Warzone2100 JavaScript Scripting API}
\date{}
\begin{document}
\maketitle

\section{Introduction}

Warzone2100 contains a scripting language for implementing AIs, campaigns and some of the game
rules. It uses JavaScript, so you should become familiar with this language before proceeding
with this document. A number of very good guides to JavaScript exist on the Internet.

The following hard-coded files exist for game rules that use this API:

\begin{description}
	\item[multiplay/skirmish/rules.js] Default game rules - base setup, starting research, winning and losing.
	\item[multiplay/script/scavfact.js] Scavenger AI. This script is active if scavengers are.
\end{description}

For ordinary AI scripts, these are controlled using '.ai' files that are present in the 'multiplayer/skirmish'
directory. Here is an example of an '.ai' file that defines an AI implemented using this API:

\begin{lstlisting}
	[AI]
	name = "Semperfi JS"
	js = semperfi.js
\end{lstlisting}

It references a '.js' JavaScript file that needs to be in the same directory as the '.ai' file. 

The code in a javascript file is accessed through specially named functions called 'events'. These are defined below. 
An event is expected to carry out some computation then return immediately. The game is on hold while an event is 
processed, so do not do too many things at once, or else the player will experience stuttering.

All global variables are saved when the game is saved. However, do not try to keep JavaScript objects that are
returned from API functions defined here around in the global scope. They are not meant to be used this way, and
bad things may happen. If you need to keep static arrays around, it is better to keep them locally defined to a
function, as they will then not be saved and loaded.

One error that it is easy to make upon initially learning JavaScript and using this API, is to try to use
the 'for (... in ...)' construct to iterate over an array of objects. This does not work! Instead, use code
like the following:

\begin{lstlisting}
var droidlist = enumDroid(me, DROID_CONSTRUCT);
for (var i = 0; i < droidlist.length; i++)
{
	var droid = droidlist[i];
	...
}
\end{lstlisting}

The above code gets a list of all your construction droids, and iterates over them one by one.

The droid object that you receive here has multiple properties that can be accessed to learn more about it. 
These propertie are read-only, and cannot be changed. In fact, objects that you get are just a copies of 
game state, and do not give any access to changing the game state itself.

Any value written in ALL_CAPS_WITH_UNDERSCORES are enums, special read-only constants defined by the
game.

\subsection{Challenges}

Challenges may load scripts as well, if a [scripts] section is present in the challenge file, and has the keys 
"extra" or "rules". The "extra" key sets up an additional script to be run during the challenge. The "rules"
key sets up an alternative rules file, which means that the "rules.js" mentioned above is \emph{not} run. In
this case, you must implement your own rules for winning and losing, among other things. Here is an example
of such a scripts section:

\begin{lstlisting}
	[scripts]
	rules = towerdefense.js
\end{lstlisting}

You can also specify which AI scripts are to be run for each player. These must be given a path to the script,
since you may sometimes want them from the AI directory ("multiplay/skirmish/") and sometimes from the challenge
directory ("challenges/"). If you do not want an AI script to be loaded for a player (for example, if you want 
this player to be controlled from one of your challenge scripts), then you can give it the special value "null". 
Here is an example if a challenge player definition with its AI specified:

\begin{lstlisting}
	[player_2]
	name = "Enemy"
	team = 1
	difficulty = "Medium"
	position = 4
	ai = multiplay/skirmish/semperfi.js
\end{lstlisting}

\section{Common Objects}

Some objects are described under the functions creating them. The following objects are produced by
multiple functions and widely used throughout, so it is better to learn about them first.

Note the special term \emph{game object} that is used in several places in this document. This refers
to the results of any function returning a Structure, Droid, Feature or Base Object as described below.
Some functions may take such objects as input parameters, in this case they may not take any other kind
of object as input parameter instead.

\input{objects}

\section{Events}

\input{events}

\section{Globals}

\begin{description}
\input{globals}
\end{description}

\section{Functions}

\input{functions}

\section{Gotchas / Bugs}

\subsection{Case sensitivity}
Due to a bug that is not so easy to fix in the short term, variables defined in global must be case insensitive.
Otherwise, they may collide on savegame loading. This is only for variables defined in your script. There is no
need to maintain case insensitivity in regards to variables defined in global by the game itself, such as 'FACTORY'
(you can safely define your own 'factory' variable) -- the only exception to this is 'me'.

\subsection{Global objects}
You must never put a \emph{game object}, such as a droid or a structure, on global. You only get a snapshot of their state,
the state is not updated, and they are not removed when they die. Trying to store them globally then using them 
later will fail.

All variables stored on global (in global scope) are stored when the game is saved, and restored when it is 
loaded. However, this may not work properly for complex objects. Basic arrays and basic types are supported,
but it is generally not recommended to put objects on global, even though simple ones may work. Since the game
can't be saved while a function is running, you don't need to worry about local variables.

Const definitions are not stored in savegames, and are therefore recommended over variables to hold information 
that does not change.

\subsection{Object states}
Most object states that are changed from the scripts are not in fact changed, but queued up to be synchronized
among clients. This means that you cannot check the state later in the same script invokation and expect it to
have changed. Instead, if for example you want to mark droids that have been ordered to do something, you can
mark them by adding a custom property. Note that this property will not be remembered when it goes out of scope.

\subsection{Early research}
You cannot set research topics for research labs directly from eventStartLevel. Instead, queue up a function
call to set it at some later frame.

\subsection{Cyborg construction}
Cyborg components are inter-linked, and cannot be passed in as lists as you can with ordinary droids, even
though they might look that way.

\end{document}

